\documentclass[a4paper,11pt,dvipdfmx]{jsarticle}


% 数式
\usepackage{amsmath,amsfonts}
\usepackage{bm}
\usepackage{physics}
\usepackage{mathtools}
% 画像
\usepackage[dvipdfmx]{graphicx}
\usepackage{circuitikz}
\usepackage{amsmath,amssymb}
\usepackage{siunitx}
\usepackage{float}
\usepackage{tikz}
\usepackage{askmaps}
\usepackage{multirow}
\usepackage{bigstrut}
\usepackage{rotating}
\usepackage{listings}
\usepackage{subcaption}
% 表
\usepackage{makecell}
% その他
\usepackage{url}
\usepackage{ascmac}
\usepackage{cases}
\usepackage{here}
\usepackage{upgreek}
\usepackage{tocloft}  % tocloftパッケージを使う
\usepackage{titlesec} % titlesecパッケージを使う(セクションタイトルのカスタマイズ)

% 画像挿入コマンド
\newcommand{\Figure}[4]{
\begin{figure}[H]
\centering
\includegraphics[width=#1\linewidth]{./images/#2}
\caption{#3}
\label{fig:#4}
\end{figure}
}

\begin{document}

% 日付とかかえてねー
\begin{table}[b]
  \centering
  \begin{tabular}{|c|c|}
    \hline
    報告者     & 22120 222 塚田 勇人 \\
    \hline
    共同実験者 & 22192 234 山本 悠介  \\ & 22060 211 古城 隆人\\
    \hline
    担当者     & 楡井 雅巳 \\
              &  藤澤 義範\\
    &力丸 彩奈\\
    &村田 雅彦\\
    &富岡 雅弘\\
    \hline
    実験年月日 & 2024年11月22日 天気:曇り 気温:21.1℃ 湿度38\verb#%#\\
    & 2024年12月6日 天気:曇り 気温:21.8℃ 湿度35\verb#%#\\
    \hline
    提出期限   & 2024年11月21日 17:00  \\
    \hline
    提出日     & \today              \\
    \hline
  \end{tabular}
\end{table}

\title{A/D変換回路の設計と製作}
\author{学籍番号:22120 \\ 組番号:222 \\名前:塚田 勇人}
\date{\today}
\maketitle

\newpage

\section{目的}
アナログ信号をデジタル信号に変換する、A/D変換回路を設計し、
製作することで、A/D変換の原理を理解する。

\section{原理}
本実験では、A/D変換回路を製作するにあたって、次の回路をユニバーサル基板
上に設計し、製作する。
\begin{itemize}
  \item 発振回路
  \item カウンタ回路
  \item ラダー回路
  \item ボルテージフォロア回路
  \item 比較回路
  \item 遅延回路
  \item ラッチ回路
  \item デコーダ回路
\end{itemize}
それぞれについて説明する。

\subsection{発振回路}
発振回路は、周波数を発生させる回路である。今回無安定マルチバイブレータ
を用いて~~Hzの周波数を発生させる。
%無安定マルチバイブレータの原理を示す。
使用した部品を\ref{tab:oscillationparts}に示す。
\begin{table}[H]
  \centering
  \caption{発振回路に使用した部品}
  \begin{tabular}{|c|c|c|c|}
    \hline
    部品名 & 型番 & 数量  \\
    \hline
    コンデンサ & 積層セラミックコンデンサ1[nF] & 2  \\
    抵抗器 & 炭素被膜抵抗300[$\Omega$]誤差$\pm$5\% & 2 \\
    抵抗器 & 炭素被膜抵抗10[k$\Omega$]誤差$\pm$5\% & 2 \\
    トランジスタ & 2SC1815 & 2  \\
    ピンヘッダ & 2 $\times$ 7 0.1[inch]ピッチ & 1  \\
    \hline
  \end{tabular}
  \label{tab:oscillationparts}
\end{table}


\subsection{カウンタ回路}


\subsection{ラダー回路}

\subsection{ボルテージフォロア回路}

\subsection{比較回路}

\subsection{遅延回路}

\subsection{ラッチ回路}

\subsection{デコーダ回路}


\section{実験環境}

\section{プログラムの設計と説明}

\section{プログラム}

\section{実行結果}

\section{考察}

\end{document}