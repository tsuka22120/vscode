\documentclass[a4paper,11pt,dvipdfmx]{jsarticle}


% 数式
\usepackage{amsmath,amsfonts}
\usepackage{bm}
\usepackage{physics}
\usepackage{mathtools}
% 画像
\usepackage[dvipdfmx]{graphicx}
\usepackage{circuitikz}
\usepackage{amsmath,amssymb}
\usepackage{siunitx}
\usepackage{float}
\usepackage{tikz}
\usepackage{askmaps}
\usepackage{multirow}
\usepackage{bigstrut}
\usepackage{rotating}
\usepackage{listings}
\usepackage{subcaption}
% 表
\usepackage{makecell}
% その他
\usepackage{url}
\usepackage{ascmac}
\usepackage{cases}
\usepackage{here}
\usepackage{upgreek}
\usepackage{tocloft}  % tocloftパッケージを使う
\usepackage{titlesec} % titlesecパッケージを使う(セクションタイトルのカスタマイズ)

% 画像挿入コマンド
\newcommand{\Figure}[4]{
\begin{figure}[H]
\centering
\includegraphics[width=#1\linewidth]{./images/#2}
\caption{#3}
\label{fig:#4}
\end{figure}
}
\begin{document}

\section{要旨}
サーミスタの温度による抵抗値の値を求めることにより,温度と抵抗値の相関を調べる.
また,結果からグラフを作成し,サーミスタの特性を確認する.

\section{目的}
今回の実験では,ホイーンストンブリッジ回路を用いて,サーミスタの温度による抵抗値の変化を調べる.
結果からグラフを作成し,温度と抵抗の相関について調べることを目的とする.
\section{実験方法}
追加資料 pp.1-4 を参照する.

\section{実験結果}
この章では測定したデータ,およびグラフを示す.
\subsection{測定データ}

\begin{table}[h]
\centering
\caption{測定データ}
\begin{tabular}{|c|c|c|c|c|c|}
\hline
No. & \( t(\SI{}{\celsius}) \) & \( T(\mathrm{K}) \) & \( 1/T \) & \( 1/T - 1/T_0 \) & \( R(\Omega) \) \\
\hline
1 & 78.0 & 351.2 & \( 2.847 \times 10^{-3} \) & \( -8.136 \times 10^{-4} \) & 1710 \\
2 & 71.0 & 344.2 & \( 2.905 \times 10^{-3} \) & \( -7.557 \times 10^{-4} \) & 2060 \\
3 & 65.2 & 338.4 & \( 2.955 \times 10^{-3} \) & \( -7.059 \times 10^{-4} \) & 2440 \\
4 & 62.8 & 336.0 & \( 2.976 \times 10^{-3} \) & \( -6.848 \times 10^{-4} \) & 2720 \\
5 & 53.8 & 327.0 & \( 3.058 \times 10^{-3} \) & \( -6.029 \times 10^{-4} \) & 3660 \\
6 & 43.6 & 316.8 & \( 3.157 \times 10^{-3} \) & \( -5.044 \times 10^{-4} \) & 5260 \\
7 & 36.5 & 309.7 & \( 3.229 \times 10^{-3} \) & \( -4.321 \times 10^{-4} \) & 6630 \\
8 & 29.8 & 303.0 & \( 3.300 \times 10^{-3} \) & \( -3.607 \times 10^{-4} \) & 8620 \\
9 & 22.4 & 295.6 & \( 3.383 \times 10^{-3} \) & \( -2.780 \times 10^{-4} \) & \( 1139 \times 10^1 \) \\
10 & 14.4 & 287.6 & \( 3.477 \times 10^{-3} \) & \( -1.839 \times 10^{-4} \) & \( 1544 \times 10^1 \) \\
11 & 8.5 & 281.7 & \( 3.550 \times 10^{-3} \) & \( -1.111 \times 10^{-4} \) & \( 1960 \times 10^1 \) \\
12 & 3.6 & 276.8 & \( 3.613 \times 10^{-3} \) & \( -4.828 \times 10^{-5} \) & \( 2340 \times 10^1 \) \\
\hline
\end{tabular}
\end{table}

tは水温,Tは絶対温度,\( T_0 \)は0℃の絶対温度,\( R \)はサーミスタの抵抗値を表す.

\subsection{グラフ}
ここでは,測定したデータをもとに作成した普通グラフと片対数グラフを示す.
\subsubsection{普通グラフ}
ここでは、測定したデータをもとに作成した普通グラフをグラフ1に示す.
\newpage
% ここに手書きのグラフを挿入する.

\subsection{片対数グラフ}
ここでは、測定したデータをもとに作成した片対数グラフをグラフ2に示す.
\newpage
% ここに手書きのグラフを挿入する.
表のデータを以下の式に代入する.
\[
\frac{1}{T} - \frac{1}{T_0}
\]
\[
y = \ln R,\quad a = \ln R_0,\quad b = B,\quad x = \frac{1}{T} - \frac{1}{T_0}
\]
と置くことで
\[
y = a + bx
\]
の形で表すことができ,

測定データと最小二乗法を用いて,値はおよそ \( a = 10.28,\ b = 3468 \) となる.

よって,傾き \( B = 3468 \),切片 \( R_0 = 2.924 \times 10^4 \) に近い値になると考えられる.

\section{考察}
本実験では,サーミスタの温度変化に伴う抵抗値の変化を測定し,得られたデータをもとに,温度と抵抗の関係をグラフ化・数式化することで,サーミスタの特性を確認した.

片対数グラフを作成し,変数変換によって \( y = \ln R,\ x = \frac{1}{T} - \frac{1}{T_0} \) とおくことで,関係式は線形 \( y = a + bx \) の形に変換された.この線形回帰から得られたパラメータは,切片 \( a = \ln R_0 \approx 10.28 \),傾き \( b = B \approx 3468 \) であり,これにより \( R_0 \approx 2.924 \times 10^4 \) と求められた.

この結果は,サーミスタの特性を表す式 \( R = R_0 \exp\left( \frac{B}{T} \right) \) の形式に一致しており,理論的にも妥当であるといえる.特に,プロットしたデータがほぼ直線上に分布していたことからも,変換後の線形関係が成り立つことが確認できた.

ただし,いくつかのデータ点では直線からのずれが見られた.この原因としては,以下のような要因が考えられる:

\begin{itemize}
    \item 水温の測定誤差:温度計の読み取りや,水中の温度むらによる影響.
    \item 抵抗測定の誤差:マルチメータの精度や接触不良による誤差.
    \item サーミスタの個体差:理想的なモデルとの乖離.
\end{itemize}

今後の改善点としては,温度の安定化のために撹拌器を用いる,より高精度な測定器を使用する,測定点数を増やすなどが挙げられる.

総じて,本実験を通してサーミスタの温度依存性を定量的に理解することができ,また,データ処理によって非線形関係を線形化し,解析を容易にする手法の有効性も確認できた.

\end{document}