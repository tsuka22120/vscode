\documentclass[a4j]{jarticle}

% 数式
\usepackage{amsmath,amsfonts}
\usepackage{bm}
% 画像
\usepackage[dvipdfmx]{graphicx}
\usepackage{listings,jvlisting}
\usepackage{jlisting}


\lstset{
basicstyle={\ttfamily},
identifierstyle={\small},
commentstyle={\smallitshape},
keywordstyle={\small\bfseries},
ndkeywordstyle={\small},
stringstyle={\small\ttfamily},
frame={tb},
breaklines=true,
columns=[l]{fullflexible},
numbers=left,
xrightmargin=0zw,
xleftmargin=3zw,
numberstyle={\scriptsize},
stepnumber=1,
numbersep=1zw,
lineskip=-0.5ex
}

% プログラミングリストコマンド
\renewcommand{\lstlistingname}{リスト}

% 画像挿入コマンド
\newcommand{\Figure}[4]{
\begin{figure}[H]
\centering
\includegraphics[width=#1\linewidth]{./images/#2}
\caption{#3}
\label{fig:#4}
\end{figure}
}
\begin{document}

\title{キュー}
\author{学籍番号:22120 \\ 組番号:222 \\名前:塚田 勇人}
\date{\today}
\maketitle
\newpage

\section{ソースコード}
\begin{lstlisting}[caption=main class,label=Queue.java]
  public class newClass {

	public static void main(String[] args) {
		Queue queue = new Queue();
		System.out.println(queue.QueueVolume() + "個目のスタックです");
		queue.enqueue(10);queue.printQueue();
		queue.enqueue(20);queue.printQueue();
		queue.enqueue(30);queue.printQueue();
		queue.enqueue(40);queue.printQueue();
		queue.enqueue(50);queue.printQueue();
		queue.enqueue(60);queue.printQueue();
		System.out.println(queue.dequeue());queue.printQueue(3,4);queue.printQueue();
		System.out.println(queue.dequeue());queue.printQueue();
		System.out.println(queue.dequeue());queue.printQueue();
		System.out.println(queue.dequeue());queue.printQueue();
		System.out.println(queue.dequeue());queue.printQueue();
		System.out.println(queue.dequeue());queue.printQueue();
	}
}
\end{lstlisting}

\begin{lstlisting}[caption=queue class class,label=Queue]

  public class Queue {
    private int volume;
    private int data[];
    private static int defaultSize = 5;
    private static int queueCount = 0;
  
    Queue() {
      this(defaultSize);
    }
  
    Queue(int n) {
      data = new int[n];
      System.out.println(data.length + "個分のキュー生成");
      queueCount++;
    }
  
    int enqueue(int number) {
      int value;
      // 残容量確認
      if (data.length > volume) {
        // 入力値確認
        if (number > 0) {
          data[volume] = number;
          volume++;
          value = 1;
        } else {
          value = 0;
          System.out.println("wrong input");
        }
      } else {
        System.out.println("queue overflow");
        value = 0;
      }
  
      return value;
    }
  
    // データ取得関数
    int dequeue() {
      int value;
      // 格納個数確認
      if (volume > 0) {
        value = data[0];
        volume--;
  
        // 空き領域を埋めるためのシフト
        for (int i = 0; i < data.length - 1; i++) {
          data[i] = data[i + 1];
        }
        data[volume] = 0;
      } else {
        value = -1;
      }
      return value;
    }
  
    // 状態表示関数
    void printQueue() {
      this.printQueue(0, data.length);
    }
  
    private void printQueue(int num){
      System.out.print(data[num]);
    }
  
    // 状態表示関数
    void printQueue(int start, int end) {
      System.out.print("|");
      for (int i = start; i < end; i++) {
        printQueue(i);
        System.out.printf("|");
      }
      System.out.println();
    }
  
    int QueueVolume() {
      return queueCount;
    }
  }
\end{lstlisting}

\section{実行結果}
\begin{lstlisting}
  5個分のキュー生成
1個目のスタックです
|10|0|0|0|0|
|10|20|0|0|0|
|10|20|30|0|0|
|10|20|30|40|0|
|10|20|30|40|50|
queue overflow
|10|20|30|40|50|
10
|50|
|20|30|40|50|0|
20
|30|40|50|0|0|
30
|40|50|0|0|0|
40
|50|0|0|0|0|
50
|0|0|0|0|0|
-1
|0|0|0|0|0|
\end{lstlisting}
\end{document}