\documentclass[a4j]{jarticle}

% 数式
\usepackage{amsmath,amsfonts}
\usepackage{bm}
% 画像
\usepackage[dvipdfmx]{graphicx}
\usepackage{listings,jvlisting}
\usepackage{jlisting}


\lstset{
basicstyle={\ttfamily},
identifierstyle={\small},
commentstyle={\smallitshape},
keywordstyle={\small\bfseries},
ndkeywordstyle={\small},
stringstyle={\small\ttfamily},
frame={tb},
breaklines=true,
columns=[l]{fullflexible},
numbers=left,
xrightmargin=0zw,
xleftmargin=3zw,
numberstyle={\scriptsize},
stepnumber=1,
numbersep=1zw,
lineskip=-0.5ex
}

% プログラミングリストコマンド
\renewcommand{\lstlistingname}{リスト}

\newcommand{\Figure}[4]{
\begin{figure}[H]
\centering
\includegraphics[width=#1\linewidth]{./images/#2}
\caption{#3}
\label{fig:#4}
\end{figure}
}

%*-------使い方--------%
%\begin{lstlisting}[caption=キャプション,label=ラベル]
% コード
%\end{lstlisting}

%\lstinputlisting[caption=キャプション,label=ラベル]{ファイル名}

%\Figure{横幅(大体0.8)}{ファイル名}{キャプション}{ラベル}

\begin{document}

\title{オブジェクト指向プログラミング}
\author{学籍番号:22120 \\ 組番号:222 \\名前:塚田 勇人}
\date{\today}
\maketitle

\section{ソースコード}
\lstinputlisting[caption=Main.java,label=src:Main.java]{./src/Main.java}
\lstinputlisting[caption=Ambulance.java,label=src:Ambulance.java]{./src/Ambulance.java}
\lstinputlisting[caption=Car.java,label=src:Car.java]{./src/Car.java}
\lstinputlisting[caption=Bus.java,label=src:Bus.java]{./src/Bus.java}
\lstinputlisting[caption=PatrolCar.java,label=src:PatrolCar.java]{./src/PatrolCar.java}
\lstinputlisting[caption=Siren.java,label=src:Siren.java]{./src/Siren.java}

\section{実行結果}
\begin{lstlisting}[caption=実行結果,label=src:result]
  ===バス===
  車製造
  バス製造
  搭載燃料:20
  タイヤ:6
  燃費:10
  ぶいーん  残燃料:10
  ぶいーん  残燃料:0
  ガス欠です!うごけません!
  ガス欠です!うごけません!
  ===救急車===
  車製造
  救急車製造
  タイヤ:4
  搭載燃料:120
  燃費:5
  ピーポーピーポー
  ===パトカー===
  車製造
  パトカー製造
  タイヤ:4
  搭載燃料:12
  燃費:5
  うおーーーーん
  ぶいーん  残燃料:7
  うおーーーーん
  ぶいーん  残燃料:2
  うおーーーーん
  ガス欠です!うごけません!
  うおーーーーん
  ガス欠です!うごけません!
\end{lstlisting}

\section{考察}
クラスの継承を用いることで、共通の機能を持つクラスを作成することができる。例えば、車のクラスを作成し、
そのクラスを継承してバス、救急車、パトカーのクラスを作成することで、車の共通の機能を持つクラスを
作成することができる。\\
また、インターフェースを用いることで、クラスに共通の機能を持たせることができる。
例えば、サイレンのインターフェースを作成し、そのインターフェースを実装することで、
サイレンの機能を持つクラスを作成することができる。
\end{document}