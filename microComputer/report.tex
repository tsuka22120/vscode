\documentclass[a4paper,11pt,dvipdfmx]{jsarticle}


% 数式
\usepackage{amsmath,amsfonts}
\usepackage{bm}
\usepackage{physics}
\usepackage{mathtools}
% 画像
\usepackage[dvipdfmx]{graphicx}
\usepackage{circuitikz}
\usepackage{amsmath,amssymb}
\usepackage{siunitx}
\usepackage{float}
\usepackage{tikz}
\usepackage{askmaps}
\usepackage{multirow}
\usepackage{bigstrut}
\usepackage{rotating}
\usepackage{listings}
\usepackage{subcaption}
% 表
\usepackage{makecell}
% その他
\usepackage{url}
\usepackage{ascmac}
\usepackage{cases}
\usepackage{here}
\usepackage{upgreek}
\usepackage{tocloft}  % tocloftパッケージを使う
\usepackage{titlesec} % titlesecパッケージを使う(セクションタイトルのカスタマイズ)

% 画像挿入コマンド
\newcommand{\Figure}[4]{
\begin{figure}[H]
\centering
\includegraphics[width=#1\linewidth]{./images/#2}
\caption{#3}
\label{fig:#4}
\end{figure}
}

\lstset{
basicstyle={\ttfamily},
identifierstyle={\small},
commentstyle={\smallitshape},
keywordstyle={\small\bfseries},
ndkeywordstyle={\small},
stringstyle={\small\ttfamily},
frame={tb},
breaklines=true,
columns=[l]{fullflexible},
numbers=left,
xrightmargin=0zw,
xleftmargin=3zw,
numberstyle={\scriptsize},
stepnumber=1,
numbersep=1zw,
lineskip=-0.5ex
}
\renewcommand{\lstlistingname}{リスト}

\begin{document}

\title{マイコンプログラミング}
\author{学籍番号:22120 \\ 組番号:222 \\名前:塚田 勇人}
\date{\today}
\maketitle

\newpage

\section{目的}
MEMESボードを用いて,簡単なゲームを作ることで,マイコンプログラミングの基礎を学ぶこと
を本課題の目的とする.

\section{実験環境}
\begin{table}
\centering
\caption{実験環境}
\label{table:environment}
\begin{tabular}{|c|c|}
\hline
OS    & Windows 11 Pro   \\
\hline
CPU   & AMD Ryzen 7 5800H \\
\hline
メモリ & 16GB              \\
\hline
コンパイラ & SH C/C++ Compiler \\
\hline
エディタ & HEW2  \\
\hline
\end{tabular}
\end{table}

\section{プログラムの設計と説明}
ここでは,本課題で追加したプログラムの仕様とその説明を行う.
追加した仕様は以下の通りである.
\begin{itemize}
  \item ゲームのスコアを7セグメントLEDに表示する
  \item ゲーム開始時にスタート音を鳴らす
  \item 岩に触れたときにミス音を鳴らす
  \item 自機をジョイスティックで左右に移動できるようにする
  \item ゲームを一時停止,再開,リセットできるようにする
\end{itemize}

\subsection{スコア表示}
スコア表示は7セグメントLEDを用いて行う.
グローバル変数scoreにスコアを格納し,\texttt{do\_7seg()}関数で7セグメントLEDに表示する.
\texttt{do\_7seg()}関数のソースコードをリスト\ref{lst:do_7seg}に示す.
この関数をゲームのメインループ内で呼び出すことで,スコアを表示する.
スコアは岩が左端に到達するごとに1点加算される.
自機が岩に触れるとスコアが5点減少する.

\begin{lstlisting}[caption=do\_7seg(), label=lst:do_7seg]
  void do_7seg() {
    static int keta;
    int num1 = score % 10;
    int num2 = (score / 10) % 10;
    int num3 = score / 100;
    PA.DR.BYTE.HL &= 0xf0;
    DIG1 = DIG2 = DIG3 = 0;

    if (keta == 0) {
        PA.DR.BYTE.HL |= num1;
        DIG1 = 1;
        keta++;
    } else if (keta == 1) {
        PA.DR.BYTE.HL |= num2;
        if (num2 == 0 && num3 == 0) {
            DIG2 = 0;
        } else {
            DIG2 = 1;
        }
        keta++;
    } else if (keta == 2) {
        PA.DR.BYTE.HL |= num3;
        if (num3 == 0) {
            DIG2 = 0;
        } else {
            DIG3 = 1;
        }
        keta = 0;
    }
}
\end{lstlisting}

\subsection{スタート音}
スタート音は\texttt{startBeep()}関数で鳴らす.
\texttt{startBeep()}関数のソースコードをリスト\ref{lst:startBeep}に示す.
この関数では,MTU2 ch3を用いて音を鳴らす.
% 周期について説明
周期$20MHz / 5000 = 4000$で用い,4kHzの音を0.1秒間で2回ならしている.


\begin{lstlisting}[caption=startBeep(), label=lst:startBeep]
  void startBeep() {
    int i = 0;

    // MTU2 ch3 sound
    MTU23.TCR.BIT.TPSC = 0;
    MTU23.TCR.BIT.CCLR = 1;
    MTU23.TGRA = 5000 - 1;
    MTU2.TSTR.BIT.CST3 = 1;

    while (i < 100) {
        if (MTU23.TSR.BIT.TGFA = 0) {
            SPK ^= 1;
            i++;
        }
    }
    wait_us(100000);
    i = 0;
    while (i < 100) {
        if (MTU23.TSR.BIT.TGFA = 0) {
            SPK ^= 1;
            i++;
        }
    }
    MTU2.TSTR.BIT.CST3 = 0;
}
\end{lstlisting}

\subsection{ミス音}

ミス音は\texttt{missBeep()}関数で鳴らす.
\texttt{missBeep()}関数のソースコードをリスト\ref{lst:missBeep}に示す.
この関数では,MTU2 ch3を用いて音を鳴らす.
% 周期について説明
周期$20MHz / 10000 = 2000$で用い,2kHzの音を0.1秒間で100回ならしている.

\begin{lstlisting}[caption=missBeep(), label=lst:missBeep]
  void missBeep() {
    int i = 0;

    // MTU2 ch3 sound
    MTU23.TCR.BIT.TPSC = 0;
    MTU23.TCR.BIT.CCLR = 1;
    MTU23.TGRA = 10000 - 1;
    MTU2.TSTR.BIT.CST3 = 1;

    while (i < 100) {
        if (MTU23.TSR.BIT.TGFA = 0) {
            SPK ^= 1;
            i++;
        }
    }
    MTU2.TSTR.BIT.CST3 = 0;
}

\end{lstlisting}

\subsection{自機の移動}
自機の移動は\texttt{move\_me()}関数で行う.
\texttt{move\_me()}関数のソースコードをリスト\ref{lst:move_me}に示す.
この関数では,ジョイスティックを横に動かすと自機も左右に移動するようにする.
また,岩に触れたときにミス音を鳴らし,スコアを減らすようにしている.

\begin{lstlisting}[caption=move\_me(), label=lst:move_me]
  void move_me(struct position *me, struct position rock[]) {
    int i;
    struct position old_position;

    old_position.x = me->x;
    old_position.y = me->y;
    me->active = 1;

    if (AD0.ADDR0 < 0x4000) {
        // -- ジョイスティック上 --
        me->y = 0;
    } else if (AD0.ADDR0 > 0xc000) {
        // -- ジョイスティック下 --
        me->y = 1;
    }
    if (AD0.ADDR1 > 0xc000 && me->x > 0) {
        me->x--;
    } else if (AD0.ADDR1 < 0x4000 && me->x < 15) {
        me->x++;
    }

    if (old_position.y != me->y || old_position.x != me->x) {
        // -- 移動したとき .. 古い表示を消す --
        LCD_cursor(old_position.x, old_position.y);
        LCD_putch(' ');
    }
    for (i = 0; i < NMROF_ROCKS; i++) {
        if (rock[i].active) {
            if ((old_position.x == rock[i].x && old_position.y == rock[i].y) ||
                (me->x == rock[i].x &&
                 me->y == rock[i].y)) {  // プレイヤーに触れたら岩を消す
                score -= 5;
                if (score < 0) score = 0;
                rock[i].active = 0;
                LCD_cursor(rock[i].x, rock[i].y);
                LCD_putch(' ');
                me->active = 0;
                missBeep();
                break;
            }
        }
    }
    if (me->active) {
        LCD_cursor(me->x, me->y);
        LCD_putch('>');
    } else {
        LCD_cursor(me->x, me->y);
        LCD_putch(' ');
    }
}
\end{lstlisting}



\section{プログラム}
プログラムをリスト\ref{lst:main}に示す.

\begin{lstlisting}[caption=main.c, label=lst:main]
  /***********************************************************************/
/*  FILE        :memes_final.c                                         */
/***********************************************************************/
#include <stdlib.h>

#include "iodefine.h"
#include "typedefine.h"

#define printf ((int (*)(const char *, ...))0x00007c7c)

#define SW6 (PD.DR.BIT.B18)
#define SW5 (PD.DR.BIT.B17)
#define SW4 (PD.DR.BIT.B16)

#define LED6 (PE.DR.BIT.B11)
#define LED_ON (0)
#define LED_OFF (1)

#define DIG1 (PE.DR.BIT.B3)
#define DIG2 (PE.DR.BIT.B2)
#define DIG3 (PE.DR.BIT.B1)

#define SPK (PE.DR.BIT.B0)

#define LCD_RS (PA.DR.BIT.B22)
#define LCD_E (PA.DR.BIT.B23)
#define LCD_RW (PD.DR.BIT.B23)
#define LCD_DATA (PD.DR.BYTE.HH)

#define NMROF_ROCKS 6

enum STS { STOP, RUN, PAUSE1, PAUSE2, PAUSE3 };

int score;

struct position {
    int x;
    int y;
    int active;
};

void wait_us(_UINT);
void LCD_inst(_SBYTE);
void LCD_data(_SBYTE);
void LCD_cursor(_UINT, _UINT);
void LCD_putch(_SBYTE);
void LCD_putstr(_SBYTE *);
void LCD_cls(void);
void LCD_init(void);
void missBeep(void);
void startBeep(void);

// --------------------
// -- 使用する関数群 --
// --------------------
void wait_us(_UINT us) {
    _UINT val;

    val = us * 10 / 16;
    if (val >= 0xffff) val = 0xffff;

    CMT0.CMCOR = val;
    CMT0.CMCSR.BIT.CMF &= 0;
    CMT.CMSTR.BIT.STR0 = 1;
    while (!CMT0.CMCSR.BIT.CMF);
    CMT0.CMCSR.BIT.CMF = 0;
    CMT.CMSTR.BIT.STR0 = 0;
}

void LCD_inst(_SBYTE inst) {
    LCD_E = 0;
    LCD_RS = 0;
    LCD_RW = 0;
    LCD_E = 1;
    LCD_DATA = inst;
    wait_us(1);
    LCD_E = 0;
    wait_us(40);
}

void LCD_data(_SBYTE data) {
    LCD_E = 0;
    LCD_RS = 1;
    LCD_RW = 0;
    LCD_E = 1;
    LCD_DATA = data;
    wait_us(1);
    LCD_E = 0;
    wait_us(40);
}

void LCD_cursor(_UINT x, _UINT y) {
    if (x > 15) x = 15;
    if (y > 1) y = 1;
    LCD_inst(0x80 | x | y << 6);
}

void LCD_putch(_SBYTE ch) { LCD_data(ch); }

void LCD_putstr(_SBYTE *str) {
    _SBYTE ch;

    while (ch = *str++) LCD_putch(ch);
}

void LCD_cls(void) {
    LCD_inst(0x01);
    wait_us(1640);
}

void LCD_init(void) {
    wait_us(45000);
    LCD_inst(0x30);
    wait_us(4100);
    LCD_inst(0x30);
    wait_us(100);
    LCD_inst(0x30);

    LCD_inst(0x38);
    LCD_inst(0x08);
    LCD_inst(0x01);
    wait_us(1640);
    LCD_inst(0x06);
    LCD_inst(0x0c);
}

// --------------------------------------------
// -- ゲーム用の関数群 --

// -- 自分を移動 --
void move_me(struct position *me, struct position rock[]) {
    int i;
    struct position old_position;

    old_position.x = me->x;
    old_position.y = me->y;
    me->active = 1;

    if (AD0.ADDR0 < 0x4000) {
        // -- ジョイスティック上 --
        me->y = 0;
    } else if (AD0.ADDR0 > 0xc000) {
        // -- ジョイスティック下 --
        me->y = 1;
    }
    if (AD0.ADDR1 > 0xc000 && me->x > 0) {
        me->x--;
    } else if (AD0.ADDR1 < 0x4000 && me->x < 15) {
        me->x++;
    }

    if (old_position.y != me->y || old_position.x != me->x) {
        // -- 移動したとき .. 古い表示を消す --
        LCD_cursor(old_position.x, old_position.y);
        LCD_putch(' ');
    }
    for (i = 0; i < NMROF_ROCKS; i++) {
        if (rock[i].active) {
            if ((old_position.x == rock[i].x && old_position.y == rock[i].y) ||
                (me->x == rock[i].x &&
                 me->y == rock[i].y)) {  // プレイヤーに触れたら岩を消す
                score -= 5;
                if (score < 0) score = 0;
                rock[i].active = 0;
                LCD_cursor(rock[i].x, rock[i].y);
                LCD_putch(' ');
                me->active = 0;
                missBeep();
                break;
            }
        }
    }
    if (me->active) {
        LCD_cursor(me->x, me->y);
        LCD_putch('>');
    } else {
        LCD_cursor(me->x, me->y);
        LCD_putch(' ');
    }
}

// -- 岩を移動 --
void move_rock(struct position rock[]) {
    int i;

    for (i = 0; i < NMROF_ROCKS; i++) {
        if (rock[i].active) {
            // 画面上に岩が存在する
            LCD_cursor(rock[i].x, rock[i].y);
            LCD_putch(' ');
            if (rock[i].x == 0) {
                // 消去
                score++;
                rock[i].active = 0;
            } else {
                rock[i].x--;
                LCD_cursor(rock[i].x, rock[i].y);
                LCD_putch('*');
            }
        }
    }
}

// -- 新しい岩を作る --
void new_rock(struct position rock[]) {
    int i;

    for (i = 0; i < NMROF_ROCKS; i++) {
        if (rock[i].active == 0) {
            // -- 新しい岩 --
            rock[i].active = 1;
            rock[i].x = 15;
            rock[i].y = rand() % 2;
            LCD_cursor(rock[i].x, rock[i].y);
            LCD_putch('*');
            break;
        }
    }
}

void do_7seg() {
    static int keta;
    int num1 = score % 10;
    int num2 = (score / 10) % 10;
    int num3 = score / 100;
    PA.DR.BYTE.HL &= 0xf0;
    DIG1 = DIG2 = DIG3 = 0;

    if (keta == 0) {
        PA.DR.BYTE.HL |= num1;
        DIG1 = 1;
        keta++;
    } else if (keta == 1) {
        PA.DR.BYTE.HL |= num2;
        if (num2 == 0 && num3 == 0) {
            DIG2 = 0;
        } else {
            DIG2 = 1;
        }
        keta++;
    } else if (keta == 2) {
        PA.DR.BYTE.HL |= num3;
        if (num3 == 0) {
            DIG2 = 0;
        } else {
            DIG3 = 1;
        }
        keta = 0;
    }
}

void missBeep() {
    int i = 0;

    // MTU2 ch3 sound
    MTU23.TCR.BIT.TPSC = 0;
    MTU23.TCR.BIT.CCLR = 1;
    MTU23.TGRA = 10000 - 1;
    MTU2.TSTR.BIT.CST3 = 1;

    while (i < 100) {
        if (MTU23.TSR.BIT.TGFA = 0) {
            SPK ^= 1;
            i++;
        }
    }
    MTU2.TSTR.BIT.CST3 = 0;
}

void startBeep() {
    int i = 0;

    // MTU2 ch3 sound
    MTU23.TCR.BIT.TPSC = 0;
    MTU23.TCR.BIT.CCLR = 1;
    MTU23.TGRA = 5000 - 1;
    MTU2.TSTR.BIT.CST3 = 1;

    while (i < 100) {
        if (MTU23.TSR.BIT.TGFA = 0) {
            SPK ^= 1;
            i++;
        }
    }
    wait_us(100000);
    i = 0;
    while (i < 100) {
        if (MTU23.TSR.BIT.TGFA = 0) {
            SPK ^= 1;
            i++;
        }
    }
    MTU2.TSTR.BIT.CST3 = 0;
}

// --------------------------------------------
// -- メイン関数 --
void main() {
    struct position me;                 // 自分の車の座標
    struct position rock[NMROF_ROCKS];  // 岩の座標
    char strPrint[16];
    int move_timing, new_timing;
    int ad, i;
    int stop_sw, run_sw, pause_sw;
    int status;

    score = 0;

    STB.CR4.BIT._AD0 = 0;
    STB.CR4.BIT._CMT = 0;
    STB.CR4.BIT._MTU2 = 0;

    CMT0.CMCSR.BIT.CKS = 1;

    // MTU2 ch0
    MTU20.TCR.BIT.TPSC = 3;   // 1/64選択
    MTU20.TCR.BIT.CCLR = 1;   // TGRAのコンペアマッチでクリア
    MTU20.TGRA = 31250 - 1;   // 100ms
    MTU20.TIER.BIT.TTGE = 1;  // A/D変換開始要求を許可

    // AD0
    AD0.ADCSR.BIT.ADM = 3;    // シングルモード
    AD0.ADCSR.BIT.CH = 1;     // AN0
    AD0.ADCSR.BIT.TRGE = 1;   // MTU2からのトリガ有効
    AD0.ADTSR.BIT.TRG0S = 1;  // TGRAコンペアマッチでトリガ

    // MTU2 ch1
    MTU21.TCR.BIT.TPSC = 3;  // 1/64選択
    MTU21.TCR.BIT.CCLR = 1;  // TGRAのコンペアマッチでクリア
    MTU21.TGRA = 31250 - 1;  // 100ms

    // MTU2 ch2 7seg
    MTU22.TCR.BIT.TPSC = 3;
    MTU22.TCR.BIT.CCLR = 1;
    MTU22.TGRA = 1000 - 1;

    LCD_init();

    MTU2.TSTR.BIT.CST0 = 1;  // MTU2 CH0スタート
    MTU2.TSTR.BIT.CST1 = 1;  // MTU2 CH1スタート
    MTU2.TSTR.BIT.CST2 = 1;  // MTU2 CH2スタート

    PFC.PAIORH.BYTE.L |= 0x0F;
    PFC.PEIORL.BIT.B1 = 1;
    PFC.PEIORL.BIT.B2 = 1;
    PFC.PEIORL.BIT.B3 = 1;

    me.x = me.y = 0;
    for (i = 0; i < NMROF_ROCKS; i++) rock[i].active = 0;

    status = STOP;
    move_timing = new_timing = 0;
    while (1) {
        if (MTU21.TSR.BIT.TGFA) {
            MTU21.TSR.BIT.TGFA = 0;

            // 100msに1回、スイッチを読む
            stop_sw = SW4;
            pause_sw = SW5;
            run_sw = SW6;

            if (status == RUN) {
                // ゲーム中
                move_me(&me, rock);  // 自分移動
                if (move_timing++ >= 2) {
                    move_timing = 0;
                    move_rock(rock);  // 岩を移動
                    if (new_timing-- <= 0) {
                        new_timing = rand() * 2 / (RAND_MAX + 1) + 1;
                        new_rock(rock);  // 新しい岩が出現
                    }
                }
                if (pause_sw) {
                    status = PAUSE1;
                    LCD_cls();
                    LCD_cursor(0, 0);
                    LCD_putstr("PAUSE");
                    LCD_cursor(0, 1);
                    LCD_putstr("PUSH switch5");
                }
                if (stop_sw) {
                    status = STOP;
                }
            } else if (status == PAUSE1) {
                if (!pause_sw)        // pause_sw が OFF なら
                    status = PAUSE2;  // status を PAUSE2 へ
            } else if (status == PAUSE2) {
                if (pause_sw)         // pause_sw が ON
                    status = PAUSE3;  // status を PAUSE3 へ
            } else if (status == PAUSE3) {
                if (!pause_sw){  // pause_sw が OFF なら
                    startBeep();
                LCD_cls();
				}
                status = RUN;  // sutatus を RUN へ
            } else { // statusがSTOPのとき
                LCD_cls();
                LCD_cursor(0, 0);
                LCD_putstr("MEMES GAMES");
                LCD_cursor(0, 1);
                LCD_putstr("PUSH switch6");
                // 停止中
                if (run_sw) {
                    startBeep();
                    status = RUN;
                    LCD_cls();
                }
            }
        }
        // if (@@@@@) として 7セグ用のタイマフラグ を見るようにするとよい
        if (MTU22.TSR.BIT.TGFA) {
            MTU22.TSR.BIT.TGFA = 0;
            do_7seg();
        }
    }
}
\end{lstlisting}

\subsection{一時停止,再開,リセット}
一時停止,再開,リセットはSW4,SW5,SW6を用いて行う.
プログラムを始めるとMEME6S GAMES Push switch6と表示される.

\section{実行結果}

\section{考察}

\end{document}